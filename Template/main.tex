\documentclass[a4paper, 12pt]{article}

%% Language and font encodings
\usepackage[english]{babel}
\usepackage[utf8x]{inputenc}
\usepackage[T1]{fontenc}
\usepackage{tabularx}
\newtheorem{definition}{Definition}

%% Sets page size and margins
\setlength{\parindent}{0pt}


%% Useful packages
\usepackage{amsmath}
\usepackage{graphicx}
\usepackage[colorinlistoftodos]{todonotes}
\usepackage[colorlinks=true, allcolors=blue]{hyperref}
\usepackage{multirow}

\title{Fysiikan toiden selostuspohja}

\begin{document}

\begin{titlepage}
    \begin{center}
        \vspace*{1cm}
 
        \textbf{HEADING}
 
        \vspace{0.5cm}
       
             
        \vspace{1.5cm}
 
        \textbf{Jussi Kujanen 273161}
        \textbf{Veikko Vittunen xxxxxx}
 
        \vfill
             
    
             
        \vspace{0.8cm}
      
   
             
       
        Tampere University\\
       Päivät tänne
             
    \end{center}
 \end{titlepage} 



% Sisollysluettelon luominen
\newpage
\thispagestyle{empty}
\tableofcontents

% Varsinainen selostus alkaa...
\newpage
\clearpage
\pagenumbering{arabic} 

\section{Definitions}
\begin{definition}
    This is a definition
\end{definition}
\section{Introduction}
The purpose of this document is to describe testing for an application which is used to sell food products. Testing is done only for the most important 
functions of the program due to time limitations. The aim of testing is to find problems that easily break the program. In addition, the purpose of this 
document is to help the reader understand how the testing and the functions to be tested have been roughly performed. 
\section{Tools}
Below this section is a description of all the tools that will be used for this project. The tools have been chosen according to our own experience and interest. 
    \subsection{Jest}
    Jest\cite{Jest} was chosen as the testframework, since it works well with React, and we both have had some experience with Jest beforehand.
    Jes is used to run the test cases described in this document. A tutorial explaining how to use jest can be found here.\cite{JestTutorial}
    \subsection{TravisCI}
    TravisCI\cite{Travis} is used during integration testing, because its automatic building and code testing during changes. It is automation provides immediate feedback which 
    is very useful even in small projects like this. It is also useful in continuous integration, since it is better to merge small code changes frequently than a large commit at 
    the end of the development. Because CI detects deficiences at the early stage in development, defects are typically less complex and smaller which makes them easier to resolve. 
    \subsection{Coveralls}
    Coveralls\cite{coveralls} is used to track test coverage. Coveralls was chosen for this task, since it is required by the course, and it is open source. 
    A tutorial on how to use Coveralls can be found here.\cite{coverallsTutorial}.
    
\section{Test cases}
With the large amount of functions present, prioritisation is a must.
Based on the description of the use case for the library, we can focus mostly on functions that handle string manipulation and validation,
such as 'capitalize','map' and 'isEmpty'. This is because the part of the frontend to be tested  is mostly based on handling user input.
\\
Functions that are not essential for string manipulation have lower priority, and are not tested.
A complete list\ref{testTable} of test cases can be found as attachment at the end of this document.


\subsection{Unit testing}
Unit tests are run using jest. For each function tested, a table is made, explaining the tests run on that function,
their inputs, outputs and expected results. 
\\
Each test is considered 'passed' once the output matches the expected output. 

The complete coverage table can be found in the attachments\ref{testTable} at the bottom of this document.
\\

\subsubsection{Test cases}

\subsection{Integration testing}

\subsubsection{Test cases}

\section{References}

  

   














            
            
           
            
            
            
            
            % Kirjallisuusviitteet ladataan referenssit.bib tiedostosta ja lohdeluettelo luodaan.
            
            \bibliographystyle{IEEEtran}
            \bibliography{referenssit}
            
            % Alkuperoinen mittauspoytokirja liitteeksi
            
            \section*{Attacments}
            
  
            \begin{table}[h]\caption{Test cases}\label{testTable}
                \begin{tabular}{|p{1.5cm}|p{2cm}|p{4cm}|p{2cm}|p{2cm}|p{1.5cm}|}    \hline
                Target file      & Test name                               & Purpose of test                                            & Test data                                      & Expected result          & Actual result \\ \hline
                1. add.js        & limit values                            & Product price uses this function                           & 0 + 5                                          & 5                        & -             \\ \hline
                2. capitalize.js & all characters capitalized              & Product descriptions must start with an upper case letter  & TESTRING                                       & Teststring               & -             \\ \hline
                3. compact.js    & large item\_id list                     & Tests large item\_id list and deletes unnecessary values   & {[}0, 1, '', 5, 6{]}                           & {[}0,1,5,6{]}            & -             \\ \hline
                4. filter.js     & filter products                         & Filter invalid product id's                                & {[}1,2,3,4,5,-5,-3{]}                          & {[}1,2,3,4,5{]}          & -             \\ \hline
                5. map.js        & similar actions for array nodes         & Multiple products need to make same actions                & {[}product1, product2{]}, capitalize         & {[}Product1, Product2{]} & -             \\ \hline
                6. slice.js      & remove certain amount from product list & Easier to delete multiple products                         & {[}1, 2, 3, 4{]} , 2 & {[}3, 4{]}               & -             \\ \hline
                7. toNumber.js   & modify string to number                 & large string list to numbers                               & '3,2'                                          & 3,2                      & -             \\ \hline
                8. toString.js   & multiple products into a single string  & Seperate products from list to a single string             & {[}p1, p2, p3, p4{]}                           & 'p1, p2, p3, p4'         & -             \\ \hline
                9. endWith.js    & Valid input                             & String validation                                          & "abc","b"                                      & False                    & -             \\ \hline
                10. endWith.js   & Invalid input                           & String validation - check for crash                        & abc,b                                          & ????                     & -             \\ \hline
                11. get.js       & Non  existent field                     & Check for crash                                        	& \{a\},'a.b'                                    & ???                      & -             \\ \hline
                12. get.js       & valid Input                             & For several keywords, might need to check every word       & \{a:b:1\},'a.b'                                & 1                        & -             \\ \hline
                13. isEmpty.js   & Null                                    & Validate empty inputs                                      & null                                           & False                    & -             \\ \hline
                14. reduce.js    & Sum of an array                         & Need to add prices of items in cart                        & {[}1,2,3{]},add,0                              & 6                        & -             \\ \hline
            \end{tabular}
            \end{table}
            
            
            \end{document}