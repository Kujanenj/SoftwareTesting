\documentclass[a4paper, 12pt]{article}

%% Language and font encodings
\usepackage[finnish]{babel}
\usepackage[utf8x]{inputenc}
\usepackage[T1]{fontenc}

%% Sets page size and margins
\setlength{\parindent}{0pt}


%% Useful packages
\usepackage{amsmath}
\usepackage{graphicx}
\usepackage[colorinlistoftodos]{todonotes}
\usepackage[colorlinks=true, allcolors=blue]{hyperref}
\usepackage{multirow}

\title{Fysiikan toiden selostuspohja}

\begin{document}

% Kansilehti
\thispagestyle{empty}

% Ylotaulukko. Muuta mittauspoivomooro, opettajan alkukirjaimet, tyon nimi seko opiskelijoiden tiedot.
\begin{tabular}{|p{2.0cm}|p{5.0cm}|p{5cm}|}
\hline 
TAU  &  FYS-1010 Fysiikan tyot 1      &  3.10.2019\\
\hline  
\multirow{2}{*}{JR}   &\multirow{2}{*}{Jousivakio}     &  284031 Mikko Rauhanen \\ &&  273161 Jussi Kujanen\\
\hline
\end{tabular}

% Sisollysluettelon luominen
\newpage
\thispagestyle{empty}
\tableofcontents

% Varsinainen selostus alkaa...
\newpage
\clearpage
\pagenumbering{arabic} 

\section{Johdanto}



\section{Teoria}



\subsection{}

  

        
        \begin{equation}\label{kaava 1}
        
        \end{equation}
        

        
        \begin{equation}\label{kaava 2}
        
        \end{equation}
        


    \subsection{}
    

        
        \begin{equation}\label{kaava 3}
        
        \end{equation}
        
        
        
        \begin{equation}\label{kaava 4}
        
        \end{equation}
        

  

    \subsection{}

    

        
        \begin{equation}\label{kaava 5}
        
        \end{equation}
        

        

        
        \begin{equation}\label{kaava 6}
        
        \end{equation}
        

    
        
        \begin{equation}\label{kaava 7}
        
        \end{equation}
        
    
    
    
        
        \begin{equation}\label{kaava 8}
        
        \end{equation}
        

    
        
        \begin{equation}\label{kaava 9}
        
        \end{equation}
        

    
        
        
        \begin{equation}\label{kaava 10}
        
        \end{equation}
        

    

        \begin{equation}\label{kaava 11}
        
        \end{equation}

  

        \begin{equation}\label{kaava 12}
           
        \end{equation}


   












\section{Mittaustulokset }






            
            \begin{figure}[h]
                \centering
                \includegraphics[width=0.8\textwidth]{}
                \caption{Jousivakio jaksonajan perusteella}
                \end{figure}
           
            \section{}
            
           
            
            
            
            \section{Yhteenveto}
            
           
            
            
            
            
            % Kirjallisuusviitteet ladataan referenssit.bib tiedostosta ja lohdeluettelo luodaan.
            
            \bibliographystyle{}
            \bibliography{}
            
            % Alkuperoinen mittauspoytokirja liitteeksi
            
            \section*{Liitteet}
            
            Liite 1: alkuperoinen mittauspöytokirjä
            
            
            
            
            \end{document}