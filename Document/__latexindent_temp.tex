\documentclass[a4paper, 12pt]{article}

%% Language and font encodings
\usepackage[english]{babel}
\usepackage[utf8x]{inputenc}
\usepackage[T1]{fontenc}
\usepackage{tabularx}

%% Sets page size and margins
\setlength{\parindent}{0pt}


%% Useful packages
\usepackage{amsmath}
\usepackage{graphicx}
\usepackage[colorinlistoftodos]{todonotes}
\usepackage[colorlinks=true, allcolors=blue]{hyperref}
\usepackage{multirow}

\title{Fysiikan toiden selostuspohja}

\begin{document}

\begin{titlepage}
    \begin{center}
        \vspace*{1cm}
 
        \textbf{HEADING}
 
        \vspace{0.5cm}
       
             
        \vspace{1.5cm}
 
        \textbf{Jussi Kujanen 273161}
        \textbf{Veikko Vittunen xxxxxx}
 
        \vfill
             
    
             
        \vspace{0.8cm}
      
   
             
       
        Tampere University\\
       Päivät tänne
             
    \end{center}
 \end{titlepage} 



% Sisollysluettelon luominen
\newpage
\thispagestyle{empty}
\tableofcontents

% Varsinainen selostus alkaa...
\newpage
\clearpage
\pagenumbering{arabic} 

\section{Definitions}

\section{Introduction}
The purpose of this document is to describe testing for an application which is used to sell food products.
\section{Tools}

    \subsection{Jest}
    Jest was chosen as the testframework, since it works well with React, and we both have had some experience with Jest beforehand.
    \subsection{TravisCI}

\section{Test cases}
With the large amount of functions present, prioritisation is a must.
Based on the description of the use case for the library, we can focus mostly on functions that handle string manipulation and validation,
such as 'capitalize','map' and 'isEmpty'.
\\
 A complete list of test cases can be found in the following chapter.


\subsection{Unit testing}

The following table shows what file is being tested, the test case name, why the test is ran, aswell as the 
input, expeted output and the actual result.
\\
The test case can be considered succesfull when the output matches the expected result.
\begin{table}[h]
    \scalebox{0.9}{
    \begin{tabular}{|p{3cm}|p{2cm}|p{4cm}|p{2cm}|p{2cm}|p{2cm}|}    \hline
        Target file      & Test name                  & Purpose of test                                           & Test data & Expected result & Actual result \\ \hline
        1. add.js        & limit values               & Product price uses this function                          & 0 + 5     & 5               & -             \\ \hline
        2. capitalize.js & all characters capitalized & Product descriptions must start with an upper case letter & TESTRING  & Teststring      & -             \\ \hline
        3. compact.js    & large item\_id list        & Tests large item\_id list and                             &           &                 &               \\ \hline
                         &                            &                                                           &           &                 &               \\ \hline
                         &                            &                                                           &           &                 &               \\ \hline
                         &                            &                                                           &           &                 &               \\ \hline
    \end{tabular}}
\end{table}


\begin{table}[]
\begin{tabular}{llllll}
\hline
Target file}      & Test name}                  & Purpose of test}                                           & Test data}         & Expected result} & Actual result} \\ \hline
1. add.js}        & limit values}               & Product price uses this function}                          & 0 + 5}             & 5}               & -}             \\ \hline
2. capitalize.js} & all characters capitalized} & Product descriptions must start with an upper case letter} & TESTRING}          & Teststring}      & -}             \\ \hline
3. compact.js}    & large item\_id list}        & Tests large item\_id list and}                             & }                  & }                & }              \\
endWith.js                             & Valid input                                     & String validation                                                              & "abc","b"                              & False                                & -                                  \\
endWith.js                             & Invalid input                                   & String validation - check for crash                                            & abc,b                                  & ????                                 & -                                  \\
get.js                                 & Non  existent field                             & Check for crash                                                                & \{a\},'a.b'                            & ???                                  & -                                  \\
get.js                                 & valid Input                                     & For several keywords, might need to check every word                           & \{a:b:1\},'a.b'                        & 1                                    & -                                  \\
isEmpty.js                             & Null                                            & Validate empty inputs                                                          & null                                   & False                                & -                                  \\ \hline
reduce.js}        & Sum of an array}            & Need to add prices of items in cart}                       & {[}1,2,3{]},add,0} & 6}               & -}            
\end{tabular}
\end{table}


\subsubsection{Test cases}

\subsection{Integration testing}

\subsubsection{Test cases}

\section{References}

  

   














            
            
           
            
            
            
            
            % Kirjallisuusviitteet ladataan referenssit.bib tiedostosta ja lohdeluettelo luodaan.
            
            \bibliographystyle{}
            \bibliography{}
            
            % Alkuperoinen mittauspoytokirja liitteeksi
            
            \section*{Liitteet}
            
  
            
            
            
            \end{document}